% note of chapter11 Model Selection and Validation, UnderstandingML

\section{Model Selection and Validation}

In this Chapter we will present two approaches for model selection.
\begin{itemize}
	\item Structural Risk Minimization;
	\item Validation.
\end{itemize}

{In this Chapter, we also consider WHAT TO DO IF LEARNING FAILS.}

\subsection{MODEL SELECTION USING SRM}

The SRM paradigm has been described and analyzed in Section 7.2.

Consider a countable sequence of hypothesis classes $\mathcal{H}_1, \mathcal{H}_2, \mathcal{H}_3, \dots$.
$\forall d$, the $\mathcal{H}_d$ enjoys the uniform convergence property
\begin{equation}
	m^{UC}_{\mathcal{H}_d}(\epsilon, \delta) \le 
	\frac{g(d)\log(1/\delta)}{\epsilon^2}
\end{equation}

We reuse $w(n) = \frac{6}{n^2 \pi^2}$ in chapter 7, we get
\begin{equation}
	\label{equ11_2}
	L_\mathcal{D} (h) \le L_S(h) + \sqrt{\frac{g(d)( \log(1/\delta) + 2\log(d) + \log(\pi^2/6) )}{m} }
\end{equation}

The upper bound given in Equation (\ref{equ11_2}) is pessimistic.

\subsection{VALIDATION}

\subsubsection{Hold Out Set}

Formally, let $V = \{ (\mathbf{x}_1, y_1), \dots, (\mathbf{x}_{m_v}, y_{m_v}) \} $ be a set of validation set.
We have
\begin{thm}
	\label{thm11_1}
	Let h be some predictor and assume that the loss function is in $[0,1]$. Then, $\forall \delta \in (0, 1)$,
	we have,
	\[ \mathbb{P} \left\{ | L_V(h)-L_\mathcal{D}(h) | \le \sqrt{ \frac{\log(2/\delta)}{2m_v} } \right\} \ge 1- \delta.\]
\end{thm}
This is tighter than the usual bounds we have seen so far. The reason for the tightness of this bound is that
it is in terms of an estimate on a fresh validation set that is independent of the way h was generated. (Compare with theorem 6.8)
\[ L_\mathcal {D} (h) \le L_S (h) + \sqrt {C \frac{d+\log(1/\delta)}{m}}.\]

For validation set can be seen as partitioning random set into two parts, so we often refer it as a \emph{hold out set}.

\subsubsection{Validation for Model Selection}

Validatiun can be naturally used for model selection.

\begin{thm}
	Let $ \mathcal{H} = {h_1, \dots, h_r}  $ be an arbitrary set of predictors and assume that the loss function is in $ [0, 1]  $ 
	be arbitrary set of predictors and assume that the loss function is in $ [0, 1]  $. Assume that a validation set V of size $ m_v $ 
	is sampled independent of $ \mathcal{H} $. Then, 
	\[ \forall h\in\mathcal{H}, \mathbb{P} \left\{ |L_\mathcal{D} - L_V(h)| \le 
	\sqrt {\frac{\log(2|\mathcal{H}|/\delta)}{2m_v}}  \right\} \ge 1-\delta.\]
\end{thm}

\subsubsection{The Model-Selection Curve}

In polynomial fitting problem, the training error is monotonically decreasing as we increase the polynomial degree.
On the other hand, the validation error first decreases but then starts to increase, 
which indicates that we are starting to suffer from overfitting.

\subsubsection{k-Fold Cross Validation}

\emph{leave-one-out} (LOO).
k-Fold cross validation is often used for model selection (or parameter tuning).

\begin{algorithm}[h!]
	\caption{k-Fold Cross Validation for Model Selection} 
	\begin{algorithmic}
		\Require{training set $S$ = $ \{ (\textbf{x}_1, y_1) , \dots, (\textbf{x}_m, y_m)  \}  $,
			set of parameter values $ \Theta $, learning algorithm $A$, integer $k$} 
		\State{\textbf{partition} S into $ S_1, S_2, \dots, S_k $ }
		\For{$ \theta \in \Theta $ }
			\For{i=1 \dots k}
				\State{$ h_i, \theta = A(S\S_i;\theta) $} 
			\EndFor.
			\State{$ error(\theta) = \frac{1}{k} \sum^k_{i=1} L_{S_i} (h_{i,\theta}) $ }
		\EndFor.
		\State{\Return$ \theta^* = \arg\min_\theta[error(\theta)], h_{\theta^*} = A(S;\theta^*) $ }
	\end{algorithmic}
\end{algorithm}

\begin{itemize}
	\item Rigorously understanding the exact behavior of cross validation is still an open problem;
	\item Rogers and Wagner have shown that for k local rules (kNN) the cross validation gives a very good estimate of the true error;
	\item Other paper show that cross validation works for stable algorithms.
\end{itemize}

\subsubsection{Train-Validation-Test Split}

In most practical applications, we split the avaliable examples into three sets.
\begin{itemize}
	\item Training set
	\item Validation set
	\item Test set
\end{itemize}

\subsection{WHAT TO DO IF LEARNING FAILS}

Main approaches for fixing:
\begin{itemize}
	\item Get a larger sample
	\item Change the hypothesis class: enlarging it; reducing it; completely changing it; changing the parameters you consider.
	\item Change the feature representation of the data
	\item Change the optimization algorithm used to apply your learning rule
\end{itemize}

\emph{Approximation error}: $ L_\mathcal{D}(h^*) $, for $ h^* \in \arg\min_{h\in\mathcal{H}} L_\mathcal{D} (h) $

\emph{Estimation error}: $ L_\mathcal{D}(h_S) - L_\mathcal{D} (h^*) $ 

Large approximation error: enlarge the hypothesis class or completely change it; change the feature representation of the data.

Large estimation error: enlarge sample set; reducing the hypothesis class.

A different error decomposition.
\[ L_\mathcal{D}(h_S) = (L_\mathcal{D}(h_S)-L_V(h_S)) + (L_V(h_S) - L_S(h_S)) + L_S(h_S).\]
\begin{itemize}
	\item $ (L_\mathcal{D}(h_S)-L_V(h_S)) $ can be bounded quite tightly using Theorem~\ref{thm11_1}.
	\item $ (L_V(h_S)-L_S(h_S)) $ is large we say that our algorithm suffers from ``overfitting''. (note good estimate)
	\item $ L_S(h_S) $ is large we say that our algorithm suffers from ``underfitting''. (not good estimate)
\end{itemize}

We write
\[ L_S(h_S) = (L_S(h_S)-L_S(h^*)) + (L_S(h^*)-L_\mathcal{D}(h^*)) + L_\mathcal{D}(h^*).\]
\begin{itemize}
	\item $ L_S(h_S)-L_S(h^*) \le 0 $ for $ ERM_\mathcal{H} $ hypothesis.
	\item $ (L_S(h_S) - L_\mathcal{D}(h^*)) $ can be bounded quite tightly (as in Theorem~\ref{thm11_1}).
	\item $ L_\mathcal{D}(h^*) $ is approximation error.
\end{itemize}
So $ L_S(h_S) $ is a necessary but not sufficient estimator for $ L_\mathcal{D}(h^*) $.
For example, when $ m < VCdim(\mathcal{H}) $, we have $ L_S(h_S) $ is 0, but $ L_\mathcal{D}(h^*) $ is high.

By learning curve:
\begin{itemize}
	\item If we see that the validation error is high and decreases with the training set but training error is zero, 
		we increase the number of examples or decrease the complexity of the hypothesis class.
	\item If the validation error is kept around 1/2 then we have no evidence that the approximation error of $ \mathcal{H} $ is good.\
		so we need increase the complexity of the hypothesis class or completely change it.
\end{itemize}

\subsection{SUMMARY}

\begin{enumerate}
	\item Plot the model-selection curve for parameter tuning.
	\item Large training error: enlarging the hypothesis class, completely change it, or change the feature representation of the data.
	\item Training error is small: plot learning curves and try to deduce from them whether the problem is estimation error or approximation error.
	\item Large estimation error and small approximation error: more training data or reducing the complexity of the hypothesis class.
	\item Large approximation error: change the hypothesis class or the feature representation of the data completely.
\end{enumerate}
